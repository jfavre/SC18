
\documentclass[]{article}
\usepackage[table]{xcolor}
\setlength\parindent{0pt}

\title{Response to the comments of Referee \# 3}

\begin{document}
\maketitle
\textbf{Referee:} This manuscript presents a study of two major different approaches, using CPUs or GPUs, for rendering large time-varying volumetric data from HPC simulations. The simulation scenario is from oceanic sheared thermal convection between cold and warm opposing sides, water surface and seafloor. The visualizations were part of the Super Computing 2018 visualization contest. As such, the manuscript is a write-up of the work conducted and the experiences and results.
The resulting visualizations are intriguing and provide valuable insights for the scientific community and the authors furthermore provide some relevant feedback to the community of visualization software developers and GPU-based rendering. \\

My recommendation for minor revision is driven primarily due the interesting topic and results but the manuscript's language needs polishing and the disposition should be slightly revised. Detailed comments follow below.
Section 2 provides context but reads more like a background and related work for a flow simulation paper. Please shorten and provide an easier to read description of the properties the visualization should be conveying. Equations seem not so important, refer the reader to existing literature if they wish to get a deeper understanding. Present the qualitatively effects various parameters have and what questions and challenges the visualizations are expected to provide answers to. Explain what the scientists would be looking for. \\

\textcolor{blue}{\textbf{Response:} Please Alexander fill in this part which concerns your section} \\

\textbf{Referee:}The last part of the last paragraph in section 2, from "We use ParaView..." should be moved and reworked into section 3. \\

\textcolor{blue}{\textbf{Response:} Indeed. We moved that at the beginning of section 3 to clearly differentiate between the numerical part (section 2), and the volume rendering setup.} \\

\textbf{Referee:} I would like to see larger images in the final print so one clearly can see them. The tables need some work as well and could be made into graphs instead for more clarity and comparison. Please don't report FPS, rather provide render times in seconds (or ms). \\

\textcolor{blue}{\textbf{Response:} We are not sure about the suggestion to use larger images, since we have to respect the two-column format. Table 3 was eliminated and replaced by a graph to allow much easier side-by-side comparisons of the startup times and average rendering times. Based also on Reviewer 1's comments, we are in fact able to add the performance numbers for the parallel implementation of the OSPRay-based rendering solution. This leads to a more balanced set of numbers, comparing all three rendering options at different scales (4, 8 and 12 compute nodes). All numbers previously expressed as Frames/sec are now expressed as rendering times.} \\

\textbf{Referee:} "is a very mature field" -> I strongly argue the opposite, DVR is well known but TFs have no clear definitions, still many unresolved questions. \\

\textcolor{blue}{\textbf{Response:} We can only agree with the reviewer that \underline{volume rendering itself} is not a mature field, and that the creation of color and opacity transfer functions is not a well-defined problem with clear-cut recipes. However, our statement in the text is about the visualization of three-dimensional scalar fields in general. We hope we can all agree on this fact. We mitigated our  comment, by stating later that volume rendering, as opposed to surface-based techniques, is “much more difficult to use” and transfer functions often defined in an “ad-hoc” manner. This is not the focus of the paper and we refer readers  to the excellent introduction in the VTK textbook.} \\

\textbf{Referee:} The reviewer offered many suggestions to improve the readability of our text, and to use a more adequate vocabulary \\

\textcolor{blue}{\textbf{Response:} Most corrections suggested have been made, and we thank the reviewer for his/her very careful reading.} \\
\end{document}
